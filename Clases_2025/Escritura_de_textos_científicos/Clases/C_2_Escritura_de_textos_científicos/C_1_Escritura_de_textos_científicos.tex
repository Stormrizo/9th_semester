\documentclass{article}
\usepackage{tikz}
\usetikzlibrary{angles, quotes}
\usepackage{amssymb}
\usepackage{pgfplots}
\pgfplotsset{width=10cm,compat=1.9}
\usepgfplotslibrary{statistics}
\usepackage[left=2.54cm,right=2.54cm,top=2.54cm,bottom=2.54cm]{geometry}
\usepackage{tasks} %Paquete necesario para la producción de items horizontales para algunos ejercicios de tarea empleando el comando /begin{multicols}}{#}.../end{multicols} para ayudarnos a reducir las páginas
\usepackage{pdflscape} %comando que permite cambiar a orientación horizontal las hojas%
\usepackage{soul} %Paquete para permitir la compilación de acentos usando el comando {\hl{}}}\\
\sethlcolor{yellow(munsell)} %comando para definir el color para el subrayado usando el comando \hl
\usepackage[utf8]{inputenc}
\usepackage[spanish]{babel}
\usepackage{icomma}
\usepackage{ragged2e}
\usepackage{siunitx}
\usepackage{url}
\usepackage[colorlinks=true, urlcolor=blue,  linkcolor=black, citecolor=green]{hyperref}
\usepackage{pdfpages}
\usepackage{blindtext} %Paquete que genera texto ficticio (dummy text) con el comando: \blindtext
\usepackage{cancel} %in the preamble gives you four different modes of striking through: \cancel{text to cancel} draws a diagonal line (slash) through its argument, \bcancel{text to cancel} uses the negative slope (a backslash), \xcancel{text to cancel} draws an X (actually \cancel plus \bcancel), \cancelto{〈value〉}{〈expression〉} draws a diagonal arrow through the 〈expression〉pointing to the 〈value〉 (math-mode only)
\usepackage{csquotes}
\usepackage{afterpage}
\usepackage{parskip} 
\usepackage{float}
\usepackage{enumitem}
\usepackage{multicol}%Paquete que permite la creación aislada de columnas de texto. De esta forma se reduce la cantidad de páginas en nuestro documento
\usepackage{lipsum} %Paquete que genera texto de relleno al igual que \usepackage{blindtext}, se usa con el comando \lipsum[#-#] los #(hashtags) delimitan el número de páginas que deseamos ocupar del paquete lipsum para usarlos como relleno. 

% Definir el comando personalizado para la nota
\newcommand{\mynote}[1]{%
\begin{tikzpicture}[baseline=-0.75ex]
	\node[draw, circle, fill=Apple Green!60, inner sep=2pt] (note) {\textbf{Nota:}};
\end{tikzpicture}%
\ \textit{#1}%
}

\newenvironment{Figura}
{\par\medskip\noindent\minipage{\linewidth}}
{\endminipage\par\medskip}
\usepackage{caption}
\usepackage[
backend=biber,
sorting=none,
url=true
]{biblatex} %bibliografía
\addbibresource{biblio.bib}
% Establecer el espacio entre las entradas de la bibliografía
\setlength{\bibitemsep}{1\baselineskip} % Puedes ajustar el valor según tus preferencias
\usepackage{amsmath,amsthm,amssymb,amsfonts}
\usepackage{pifont} %Permite la compilación del comando \xmark: es el símbolo de la tachita
\usepackage{mathtools} %permite la compilación de símbolos para matrices
\usepackage{empheq} %Paquete que se relaciona con \usepackage[most]{tcolorbox} para la creación de cuadros/cajas de colores para delimitar los resultados matemáticos o líneas de texto usando la línea de comando: \begin{empheq}[box={\mymath[colback=orange(webcolor)!70,drop lifted shadow]}]{equation*} \end{empheq}
\usepackage[most]{tcolorbox} %Paquete que se relaciona con \usepackage{empheq} para la creación de cuadros/cajas de colores para delimitar los resultados matemáticos o líneas de texto
\renewcommand{\qedsymbol}{$\blacksquare$}
\usetikzlibrary{positioning,decorations.pathreplacing} %Paquete que permite la compilación de llaves para cuadros sinópticos (brace diagram) 
\usepackage{schemata} %The schemata package is designed just for these brace diagrams
\usepackage{booktabs}

\newcommand\AB[2]{\schema{\schemabox{#1}}{\schemabox{#2}}} %comando o paquete necesario para crear cuadros sinópticos

\newtcbox{\mymath}[1][]{%
nobeforeafter, math upper, tcbox raise base,
enhanced, colframe=blue!30!black,
colback=blue!30, boxrule=1pt,
#1} %Comando importante para encasillar los resultados matemáticos en cajas de diferentes colores, se relaciona con los paquetes \usepackage{empheq} y \usepackage[most]{tcolorbox} 

\newenvironment{sysmatrix}[1]
{\left(\begin{array}{@{}#1@{}}}
{\end{array}\right)}
\newcommand{\ro}[1]{%
\xrightarrow{\mathmakebox[\rowidth]{#1}}%
}
\newlength{\rowidth}% row operation width
\AtBeginDocument{\setlength{\rowidth}{3em}}  %Comando importante para laproducción de lineas de operaciones entre matrices, método de Gauss o Gauss Jordan se relaciona con \newenvironment{sysmatrix}

%formato para cambiar el horario a español
\usepackage[spanish]{datetime2}
\DTMsetdatestyle{spanish}

\renewcommand{\today}{\DTMdisplaydate{\the\year}{\the\month}{\the\day }{-1}}
%%%%%%%%%%%%%%%%%%%%%%%%%%%%%%%%%%%%%%%%%

\usepackage{datetime}
\newcommand{\mycurrenttime}{\xxivtime}

%%%%%%%%%%%%%%%%%%%%%%%%%%%%%%%%%%%%%%%%%%%%%%%%%%%%%%%%%%%%%%%%%%%%%%%%%%%%%%%%%%%%%%%%%%%%%%%%%%%%%%%%%%%%%%%%%%%%%%%%%%%%%%%%%%%%%%%%%%%%%%%%%


%%%%%%%%%%%%%Caja de color para el título%%%%%%%%%%%%%%%%%%%%%%%%%%%%%%%%%%%%%%%%%%%%%%%%%%%%%%

\definecolor{myframecolor}{RGB}{1, 45, 75} % Prussian Blue
\definecolor{myboxcolor}{RGB}{0, 107, 92} % Apple Green

\newtcolorbox{mybox}{
enhanced,
colback=myboxcolor!7, % Color de fondo del cuadro
colframe=myframecolor, % Color del marco del cuadro
arc=0pt,
boxrule=1pt,
borderline west={2mm}{-10mm}{myframecolor}, % Borde en el lado izquierdo
sharp corners=southwest,
width=\linewidth,
before=\par\vspace{\bigskipamount}, % Espacio antes del cuadro
after=\par\vspace{\bigskipamount} % Espacio después del cuadro
}


\newcommand{\euler}{e} %Comando para producir letra e de euler
\newenvironment{remark}{\par\vfill\footnotesize % Vertical white space above the remark and smaller font size
\begin{list}{}{
		\leftmargin=80pt % Indentation on the left
		\rightmargin=60pt}\item\ignorespaces % Indentation on the right
	\makebox[-2.5pt]{\begin{tikzpicture}[overlay]
			\node[draw=Horizon!60,line width=2.5pt,circle,fill=Horizon!25,font=\sffamily\bfseries,inner sep=4pt,outer sep=0pt] at (-19pt,5pt){\textcolor{Horizon}{Nota}};\end{tikzpicture}} % Blue Nota in a circle
	\advance\baselineskip -1pt}{\end{list}\vskip5pt} % Tighter line spacing and white space after remark
	\usepackage{graphicx}
	
	\usepackage{titling}
	
	\input{colores.tex}
	
	%%Producir signos de cita textual``Asignatura''
	
	\newcommand{\xmark}{\ding{55}} %%%comando que genera la tachita
	
	\newenvironment{MyColorPar}[1]{%
\leavevmode\color{#1}\ignorespaces%
}{%
}%

%%%%%%%%%%%%%%%%%%% Título de la tarea, Nombre de alumno

\title{ \textcolor{Tarawera}{\textbf{T-\textcolor{Sun}{\textbf{2}}}}: \textcolor{Cinnabar}{\textbf{Problemas}}}
\author{\emph{Julio Alfredo Ballinas García} $\boldsymbol{\mid}$ 202107583}
\date{\today}

%%%%%%%%%%%%%%%%%%% Datos de la Materia

\usepackage{fancyhdr}
\fancypagestyle{plain}{%  the preset of fancyhdr 
\fancyhf{} % clear all header and footer fields
\fancyfoot[R]{\includegraphics[width=2cm]{LogoFCFMBUAP (1).png}}
\fancyfoot[L]{{\bfseries{\thedate{}}} a las {\bfseries{\mycurrenttime{}}} horas{} (GMT-6, H. Puebla de Zaragoza, Pue)}
\fancyhead[L]{Mecánica teórica ({\bfseries{N.R.C}}) (FISS-257) }
\fancyhead[R]{\theauthor}
}
\makeatletter
\def\@maketitle{%
\newpage
\null
\vskip 1em%
\begin{center}%
	\let \footnote \thanks
	{\LARGE \@title \par}%
	\vskip 1em%
	%{\large \@date}%
\end{center}%
\par
\vskip 1em}
\makeatother

\usepackage{lipsum}  
\usepackage{cmbright}


%%%%%%%%%%%%%%%%%%%%%%%%%%%%%%%%%%%%%%%%%%%%%%%%%%%%%%%%%%%%%%%%%%%%%%%%%%%%%%%%%%%%%%%%%%%%%%%%%%%%%%%%%%%%%%%%%%%%%%%%%%%%%%%%%%%%%%%%%%%%%%%%%%%%%%%%%%%%%%%%%%%%%%%%%%%%%%%%%%%%%%%%%%%%%%%%%%%%%%

\begin{document}



\maketitle


%%%%%%%%%%%%%%%%%%% Datos del profesor, campus y aula

\begin{mybox}
	\noindent\begin{tabular}{@{}ll}
		{\bfseries{Profesor}} & Dr. Alberto Escalante Hernández\\
		\textcolor{prussianblue}{\bfseries{Campus}} / \textcolor{trueblue}{\bfseries{Facultad}}  & \textcolor{prussianblue}{\bfseries{C.U. BUAP}} /  \textcolor{trueblue}{\bfseries{FCFM}} \\
		\textcolor{prussianblue}{\bfseries{Edificio}} /  \textcolor{trueblue}{\bfseries{Salón}}    & 	\textcolor{prussianblue}{\bfseries{1FM2}} /  \textcolor{trueblue}{\bfseries{301B}}   
	\end{tabular} 
\end{mybox} \vspace{1cm}

\section*{Instrucciones de los ejercicios}
\addcontentsline{toc}{section}{Instrucciones de los ejercicios}

\setlength{\columnsep}{2cm}
\begin{multicols}{2} % Comienza las dos columnas
	%Comienza el primer entorno enumerate
	\begin{enumerate}[label={{\textcolor{trueblue}{\textbf{Ej}. \arabic*.0}}}, start=1]
		\item Mostar que la trayectoria más corta sobre la superficie de un cílindro circular recto es una hélice.
	\end{enumerate}
	
	
	\begin{enumerate}[label={{\textcolor{trueblue}{\textbf{Ej}. \arabic*.0}}}, start=2]
		\item Encontrar la trayectoria más corta entre dos puntos que están sobre una esfera.
	\end{enumerate}  
	
	\begin{enumerate}[label={{\textcolor{trueblue}{\textbf{Ej}. \arabic*.0}}}, start=3]
		\item Encontrar la trayectoria más corta entre los puntos $(0,-1,0)$ y $(0,1,0)$ sobre la superficie $z=1-\sqrt{x^{2}+y^{2}}$ (\textbf{sugerencia:} usar coordenadas \underline{cilíndricas}.)
	\end{enumerate} 
	
	\begin{enumerate}[label={{\textcolor{trueblue}{\textbf{Ej}. \arabic*.0}}}, start=5]
		\item A particle is subjected to the potential $V(x) = -Fx$ where $F$ is a constant. The particle travels from $x=0$ to $x=a$ in a time interval $t_{0}$. Assume the motion of the particle can be expressed in the form $x(t)=A + Bt + Ct^{2}$. Find the values of $A$, $B$, and $C$ such that the action is a minimum
	\end{enumerate} 
	
	\begin{enumerate}[label={{\textcolor{trueblue}{\textbf{Ej}. \arabic*.0}}}, start=12]
		\item The term \textit{generalized mechanics} has come to designate a variety of classical mechanics in which the Langrangian contains time derivatives of $q_{i}$ higher than the first. Problems for which $\dddot{x} = f(x,\dot{x}, \ddot{x}, t)$ have been referred to as \textbf{jerky} mechanics. Such equations of motions have interesting applications in chaos theory (cf. Chapter 11). By applying the methods of the calculus of variations, show that if there is a Lagrangian of the form $L\left( q_{i}, \dot{q}_{i}, \ddot{q}_{i}, t \right)$, and Hamilton´s principle holds with the zero variation of both $q_{i}$ and $\dot{q}_{i}$ at the end points, then corresponding Euler-Lagrange equations are:
		
		\[  \frac{ d^{2} }{dt} \left(  \frac{ \partial L }{ \partial \ddot{q}_{i} } \right) - \frac{d}{dt} \left(  \frac{ \partial L  }{\partial \dot{q}_{i}   }  \right) + \frac{\partial L}{\partial q_{i}} =0 \quad i = 1,2,...,n  \]
		
		Apply this result to the Langranian:
		
		\[ L = - \frac{m}{2}q\ddot{q}-\frac{k}{2}q^{2} \]
		
		Do you recognize the equations of motion?
	\end{enumerate} 
	
\begin{enumerate}[label={{\textcolor{trueblue}{\textbf{Ej}. \arabic*.0}}}, start=14]
		\item A uniform hoop of mass $m$ and radius $r$ rolls without slipping on a fixed cylinder of radius $R$ as shown in the figure. The only external force is that of gravity. If the smaller cylinder starts rolling from rest on top of the bigger cylinder, use the method of Lagrange multipliers to find the point at which the hoop falls off the cylinder.
\end{enumerate} 



\begin{enumerate}[label={{\textcolor{trueblue}{\textbf{Ej}. \arabic*.0}}}, start=16]
	\item In certain situations, particularly one-dimensional systems, it is possible to incorporate frictional effects without introducing the dissipation function. As an example, find the equations of motion for the Lagrangian
	
	\[  L = e^{\gamma{t}} \left(  \frac{ m\dot{q}^{2}  }{2} - \frac{ kq^{2}  }{ 2  }      \right)        \].
	
	How would you describe the  system? Are there any constants of motion? Suppose a point transformation is made of the form
	
	\[ s = e^{\frac{\gamma{t}}{2} }q \]
	
	What is the effective Lagrangian in terms of $s$? Find the equation of motion for $s$. What do these results say about the conserved quantities for the system?  
	
	
\end{enumerate}
	
\end{multicols} 

\newpage


\tableofcontents \newpage

\begin{center}
	\section*{\textcolor{Cinnabar}{\bfseries{Soluciones}}}
	\addcontentsline{toc}{section}{\textcolor{Cinnabar}{\bfseries{Soluciones}} \hspace{1mm} \textcolor{bittersweet}{$\bullet$} \textcolor{Apple Green}{$\bullet$} \textcolor{trueblue}{$\bullet$} \textcolor{Aguamarina}{$\bullet$} \textcolor{Sun}{$\bullet$} \textcolor{orange(colorwheel)}{$\bullet$} \textcolor{Pesto}{$\bullet$} \textcolor{blue-violet}{$\bullet$} \textcolor{ticklemepink}{$\bullet$} \textcolor{Gallery}{$\bullet$} \textcolor{Mantle}{$\bullet$} \textcolor{Nevada}{$\bullet$} \textcolor{onyx}{$\bullet$}  } 
\end{center}  \vspace{0.5cm}


\begin{center}
	\begin{tikzpicture}
		% Dibuja círculos rellenos en diferentes colores
		\fill[bittersweet] (0,0) circle (0.17cm);
		\fill[Apple Green] (0.6,0) circle (0.17cm);
		\fill[trueblue] (1.2,0) circle (0.17cm);
		\fill[Aguamarina] (1.8,0) circle (0.17cm);
		\fill[Sun] (2.4,0) circle (0.17cm);
		\fill[orange(colorwheel)] (3.0,0) circle (0.17cm);
		\fill[Pesto] (3.6,0) circle (0.17cm);
		\fill[blue-violet] (4.2,0) circle (0.17cm);
		\fill[ticklemepink] (4.8,0) circle (0.17cm);
		\fill[Gallery] (5.4,0) circle (0.17cm);
		\fill[Mantle] (6.0,0) circle (0.17cm);
		\fill[Nevada] (6.6,0) circle (0.17cm);
		\fill[onyx] (7.2,0) circle (0.17cm);
		% Agrega más círculos o modifica los colores según sea necesario
	\end{tikzpicture}
\end{center} \vfill

\section{Clase {\#} 2 } \vspace{0.5cm}


\subsection{Técnicas de parafraseo} \vspace{0.5cm}

Vamos a cambiar la función gramtical de algunas palabras. \vspace{0.5cm}

\subsubsection{Texto de ejemplo} \vspace{0.5cm}


Lo qu queremos hacer es como funciona la grámatica. Decimos lo mismo, pero con otras palabras


\textbf{Original}: El profesor de medicina \underline{John Swason} plantea que los cambios globales influyen en la propagación de las enfermedades. \vspace{0.5cm}


\textbf{interpretación}:  John Swason, profesor de medicina sugiere que al analizar enfermedades globales, si existen cambios en los parámetros mundiales, entonces estos mismos cambios influyen en la propagación de estas enfermedades. 


\textbf{Bloqueo del escritor} : No saber como entrarle al artículo. 

\textbf{Parafrasis}: De acuerdo con John Swason, profesor de medidicna, los camnbios del mundo causa que las enfermedades se propague. (James, 2004).

Formato APA.

Generalmete en la ciencia se toman los lineamientos de la triple \textbf{EEE}.

Eso faltaba, la paráfrasis.

Cambiar la función gramatical de algunas palabras. 

Siempre incluir de donde tomamos la información. 


\subsubsection{Segunda técnica} \vspace{0.5cm}

\textbf{Uso de sinónimos} \vspace{0.5cm}

Si incorporamos algo al párrafo de nuestro interés, tenemos que citar al autor.

No se trata tampoco de citar a 1000 autores, si el campo de investigación es muy explorada, es probable que existan autores top. Otro puede ser artículos de revisión, alguien ya hizo el trabajo de condensar las cosas.

Debemos usar sinónimos, es decir ampliar el vocabularios, para decir los mismo, pero cambiando las palabras. 

\textbf{Otro ejemplo}:

\textbf{ORIGINAL}: Un portavoz del gobierno norteamericano declaró que la crisis por pacientes con cáncer representa una amenaza a la seguridad nacional. El anuncio se hizo después que, en un informe de inteligencia, se asegura que las altas tasas de cáncer podría provocar que se generalice una desestabilización política.

\textbf{Interpretación usando sinónimos}: Un representante del gobierno de Estados Unidos ha declarado que la crisis con pacientes con cáncer es un factor de riesgo para la seguridad nacional. Esta declaración fue realizada luego de un informe de inteligencia en donde se manifiesta que las altas incidencias de cáncer pueden desestabilizar el entorno político. 


\textbf{Se podrían usar un poco más de sinónimos.} \vspace{0.5cm}

\textbf{Paráfrasis}: Un portavoz del gobierno de los Estados Unidos anunció que diversas variedades de cáncer podrían poner en peligro la seguridad de esa nación. El gobierno alertó $\ldots$ \vspace{0.5cm}

\underline{cambiar la función gramatical de algunas palabras} \vspace{0.5cm}

\textbf{ORIGINAL}: Los grupos minoritarios de los Estados Unidos han sido golpeados con más fuerza por la epidemia. Los afroamericanos, que constituyen el 13 por ciento de la población norteamericana, representaban el 46 por ciento de los caso de SIDA en 1998.     


\textbf{Interpretación cambiando la función gramatical de algunas palabras}: Los afroamericanos, que representan un grupo minoritario del alrededor del 13 por ciento de la población norteamericana son unos de los grupos más golpeados por la epidemia de SIDA. Representaban el 46 por ciento de los casos en 1998. 

Vemos que está escrito un acrónimo (\underline{SIDA}). No están escribiendo todo el nombre, si usamos un acrónimo. La primera palabra del acrónimo la tenemos que definir. Por ejemplo, en el caso del dengue, dengue virus, luego el acrónimo. DN-P. A parti de ahí se puede utilizar el acrónimo, sin embargo se tiene que definir una vez la palabra. \vspace{0.5cm}


\textbf{Paráfrasis}: La epidemia de SIDA ha afectado principalmente $\ldots$ \vspace{0.5cm}
 
 Luego al final va la cita textual, al final va la cita de esta paráfrasis. 
 
 El asunto de la paráfrasis, podemos modificar de muchas maneras. Cambiar el orden de las palabras (p. ej. cambiar de voz activa a pasiva y mover los modificadores a diferentes posiciones.)
 
 \textbf{ORIGINAL}: Angier (2001) informó que la malaria mata a más de un millón de personas anualmente, la gran mayoría de ellos son niños de la región subsahariana. 
 
 \textbf{Interpretación}: Según Angier (2001) la gran mayoría de personas que han perecido debido a la malaria son principalmente niños de la región subsahariana y representa más de un millón de casos anualmente. 
 
 
 \textbf{PARAFRASIS}: Cada año, más de un millón de personas muere a causa de la malaria, y la mayoría ed las víctimas son niños que viven en el África subsahariana (Angier, 2001).
 
 
 Actividad, de nuevo realizar la interpretación de los ejemplos que la Dra. nos manda los ORIGINALES.
 La actividad anterior a más tardar para mañana.
 
 Investigar para el día martes que es un \textbf{congreso de investigación} escribir un resumen de un congreso.  


  


\end{document}
