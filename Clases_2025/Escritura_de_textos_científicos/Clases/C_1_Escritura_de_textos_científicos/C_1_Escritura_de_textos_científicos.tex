\documentclass{article}
\usepackage{tikz}
\usetikzlibrary{angles, quotes}
\usepackage{amssymb}
\usepackage{pgfplots}
\pgfplotsset{width=10cm,compat=1.9}
\usepgfplotslibrary{statistics}
\usepackage[left=2.54cm,right=2.54cm,top=2.54cm,bottom=2.54cm]{geometry}
\usepackage{tasks} %Paquete necesario para la producción de items horizontales para algunos ejercicios de tarea empleando el comando /begin{multicols}}{#}.../end{multicols} para ayudarnos a reducir las páginas
\usepackage{pdflscape} %comando que permite cambiar a orientación horizontal las hojas%
\usepackage{soul} %Paquete para permitir la compilación de acentos usando el comando {\hl{}}}\\
\sethlcolor{yellow(munsell)} %comando para definir el color para el subrayado usando el comando \hl
\usepackage[utf8]{inputenc}
\usepackage[spanish]{babel}
\usepackage{icomma}
\usepackage{ragged2e}
\usepackage{siunitx}
\usepackage{url}
\usepackage[colorlinks=true, urlcolor=blue,  linkcolor=black, citecolor=green]{hyperref}
\usepackage{pdfpages}
\usepackage{blindtext} %Paquete que genera texto ficticio (dummy text) con el comando: \blindtext
\usepackage{cancel} %in the preamble gives you four different modes of striking through: \cancel{text to cancel} draws a diagonal line (slash) through its argument, \bcancel{text to cancel} uses the negative slope (a backslash), \xcancel{text to cancel} draws an X (actually \cancel plus \bcancel), \cancelto{〈value〉}{〈expression〉} draws a diagonal arrow through the 〈expression〉pointing to the 〈value〉 (math-mode only)
\usepackage{csquotes}
\usepackage{afterpage}
\usepackage{parskip} 
\usepackage{float}
\usepackage{enumitem}
\usepackage{multicol}%Paquete que permite la creación aislada de columnas de texto. De esta forma se reduce la cantidad de páginas en nuestro documento
\usepackage{lipsum} %Paquete que genera texto de relleno al igual que \usepackage{blindtext}, se usa con el comando \lipsum[#-#] los #(hashtags) delimitan el número de páginas que deseamos ocupar del paquete lipsum para usarlos como relleno. 

% Definir el comando personalizado para la nota
\newcommand{\mynote}[1]{%
\begin{tikzpicture}[baseline=-0.75ex]
	\node[draw, circle, fill=Apple Green!60, inner sep=2pt] (note) {\textbf{Nota:}};
\end{tikzpicture}%
\ \textit{#1}%
}

\newenvironment{Figura}
{\par\medskip\noindent\minipage{\linewidth}}
{\endminipage\par\medskip}
\usepackage{caption}
\usepackage[
backend=biber,
sorting=none,
url=true
]{biblatex} %bibliografía
\addbibresource{biblio.bib}
% Establecer el espacio entre las entradas de la bibliografía
\setlength{\bibitemsep}{1\baselineskip} % Puedes ajustar el valor según tus preferencias
\usepackage{amsmath,amsthm,amssymb,amsfonts}
\usepackage{pifont} %Permite la compilación del comando \xmark: es el símbolo de la tachita
\usepackage{mathtools} %permite la compilación de símbolos para matrices
\usepackage{empheq} %Paquete que se relaciona con \usepackage[most]{tcolorbox} para la creación de cuadros/cajas de colores para delimitar los resultados matemáticos o líneas de texto usando la línea de comando: \begin{empheq}[box={\mymath[colback=orange(webcolor)!70,drop lifted shadow]}]{equation*} \end{empheq}
\usepackage[most]{tcolorbox} %Paquete que se relaciona con \usepackage{empheq} para la creación de cuadros/cajas de colores para delimitar los resultados matemáticos o líneas de texto
\renewcommand{\qedsymbol}{$\blacksquare$}
\usetikzlibrary{positioning,decorations.pathreplacing} %Paquete que permite la compilación de llaves para cuadros sinópticos (brace diagram) 
\usepackage{schemata} %The schemata package is designed just for these brace diagrams
\usepackage{booktabs}

\newcommand\AB[2]{\schema{\schemabox{#1}}{\schemabox{#2}}} %comando o paquete necesario para crear cuadros sinópticos

\newtcbox{\mymath}[1][]{%
nobeforeafter, math upper, tcbox raise base,
enhanced, colframe=blue!30!black,
colback=blue!30, boxrule=1pt,
#1} %Comando importante para encasillar los resultados matemáticos en cajas de diferentes colores, se relaciona con los paquetes \usepackage{empheq} y \usepackage[most]{tcolorbox} 

\newenvironment{sysmatrix}[1]
{\left(\begin{array}{@{}#1@{}}}
{\end{array}\right)}
\newcommand{\ro}[1]{%
\xrightarrow{\mathmakebox[\rowidth]{#1}}%
}
\newlength{\rowidth}% row operation width
\AtBeginDocument{\setlength{\rowidth}{3em}}  %Comando importante para laproducción de lineas de operaciones entre matrices, método de Gauss o Gauss Jordan se relaciona con \newenvironment{sysmatrix}

%formato para cambiar el horario a español
\usepackage[spanish]{datetime2}
\DTMsetdatestyle{spanish}

\renewcommand{\today}{\DTMdisplaydate{\the\year}{\the\month}{\the\day }{-1}}
%%%%%%%%%%%%%%%%%%%%%%%%%%%%%%%%%%%%%%%%%

\usepackage{datetime}
\newcommand{\mycurrenttime}{\xxivtime}

%%%%%%%%%%%%%%%%%%%%%%%%%%%%%%%%%%%%%%%%%%%%%%%%%%%%%%%%%%%%%%%%%%%%%%%%%%%%%%%%%%%%%%%%%%%%%%%%%%%%%%%%%%%%%%%%%%%%%%%%%%%%%%%%%%%%%%%%%%%%%%%%%


%%%%%%%%%%%%%Caja de color para el título%%%%%%%%%%%%%%%%%%%%%%%%%%%%%%%%%%%%%%%%%%%%%%%%%%%%%%

\definecolor{myframecolor}{RGB}{1, 45, 75} % Prussian Blue
\definecolor{myboxcolor}{RGB}{0, 107, 92} % Apple Green

\newtcolorbox{mybox}{
enhanced,
colback=myboxcolor!7, % Color de fondo del cuadro
colframe=myframecolor, % Color del marco del cuadro
arc=0pt,
boxrule=1pt,
borderline west={2mm}{-10mm}{myframecolor}, % Borde en el lado izquierdo
sharp corners=southwest,
width=\linewidth,
before=\par\vspace{\bigskipamount}, % Espacio antes del cuadro
after=\par\vspace{\bigskipamount} % Espacio después del cuadro
}


\newcommand{\euler}{e} %Comando para producir letra e de euler
\newenvironment{remark}{\par\vfill\footnotesize % Vertical white space above the remark and smaller font size
\begin{list}{}{
		\leftmargin=80pt % Indentation on the left
		\rightmargin=60pt}\item\ignorespaces % Indentation on the right
	\makebox[-2.5pt]{\begin{tikzpicture}[overlay]
			\node[draw=Horizon!60,line width=2.5pt,circle,fill=Horizon!25,font=\sffamily\bfseries,inner sep=4pt,outer sep=0pt] at (-19pt,5pt){\textcolor{Horizon}{Nota}};\end{tikzpicture}} % Blue Nota in a circle
	\advance\baselineskip -1pt}{\end{list}\vskip5pt} % Tighter line spacing and white space after remark
	\usepackage{graphicx}
	
	\usepackage{titling}
	
	\input{colores.tex}
	
	%%Producir signos de cita textual``Asignatura''
	
	\newcommand{\xmark}{\ding{55}} %%%comando que genera la tachita
	
	\newenvironment{MyColorPar}[1]{%
\leavevmode\color{#1}\ignorespaces%
}{%
}%

%%%%%%%%%%%%%%%%%%% Título de la tarea, Nombre de alumno

\title{ \textcolor{Tarawera}{\textbf{T-\textcolor{Sun}{\textbf{2}}}}: \textcolor{Cinnabar}{\textbf{Problemas}}}
\author{\emph{Julio Alfredo Ballinas García} $\boldsymbol{\mid}$ 202107583}
\date{\today}

%%%%%%%%%%%%%%%%%%% Datos de la Materia

\usepackage{fancyhdr}
\fancypagestyle{plain}{%  the preset of fancyhdr 
\fancyhf{} % clear all header and footer fields
\fancyfoot[R]{\includegraphics[width=2cm]{LogoFCFMBUAP (1).png}}
\fancyfoot[L]{{\bfseries{\thedate{}}} a las {\bfseries{\mycurrenttime{}}} horas{} (GMT-6, H. Puebla de Zaragoza, Pue)}
\fancyhead[L]{Mecánica teórica ({\bfseries{N.R.C}}) (FISS-257) }
\fancyhead[R]{\theauthor}
}
\makeatletter
\def\@maketitle{%
\newpage
\null
\vskip 1em%
\begin{center}%
	\let \footnote \thanks
	{\LARGE \@title \par}%
	\vskip 1em%
	%{\large \@date}%
\end{center}%
\par
\vskip 1em}
\makeatother

\usepackage{lipsum}  
\usepackage{cmbright}


%%%%%%%%%%%%%%%%%%%%%%%%%%%%%%%%%%%%%%%%%%%%%%%%%%%%%%%%%%%%%%%%%%%%%%%%%%%%%%%%%%%%%%%%%%%%%%%%%%%%%%%%%%%%%%%%%%%%%%%%%%%%%%%%%%%%%%%%%%%%%%%%%%%%%%%%%%%%%%%%%%%%%%%%%%%%%%%%%%%%%%%%%%%%%%%%%%%%%%

\begin{document}



\maketitle


%%%%%%%%%%%%%%%%%%% Datos del profesor, campus y aula

\begin{mybox}
	\noindent\begin{tabular}{@{}ll}
		{\bfseries{Profesor}} & Dr. Alberto Escalante Hernández\\
		\textcolor{prussianblue}{\bfseries{Campus}} / \textcolor{trueblue}{\bfseries{Facultad}}  & \textcolor{prussianblue}{\bfseries{C.U. BUAP}} /  \textcolor{trueblue}{\bfseries{FCFM}} \\
		\textcolor{prussianblue}{\bfseries{Edificio}} /  \textcolor{trueblue}{\bfseries{Salón}}    & 	\textcolor{prussianblue}{\bfseries{1FM2}} /  \textcolor{trueblue}{\bfseries{301B}}   
	\end{tabular} 
\end{mybox} \vspace{1cm}

\section*{Instrucciones de los ejercicios}
\addcontentsline{toc}{section}{Instrucciones de los ejercicios}

\setlength{\columnsep}{2cm}
\begin{multicols}{2} % Comienza las dos columnas
	%Comienza el primer entorno enumerate
	\begin{enumerate}[label={{\textcolor{trueblue}{\textbf{Ej}. \arabic*.0}}}, start=1]
		\item Mostar que la trayectoria más corta sobre la superficie de un cílindro circular recto es una hélice.
	\end{enumerate}
	
	
	\begin{enumerate}[label={{\textcolor{trueblue}{\textbf{Ej}. \arabic*.0}}}, start=2]
		\item Encontrar la trayectoria más corta entre dos puntos que están sobre una esfera.
	\end{enumerate}  
	
	\begin{enumerate}[label={{\textcolor{trueblue}{\textbf{Ej}. \arabic*.0}}}, start=3]
		\item Encontrar la trayectoria más corta entre los puntos $(0,-1,0)$ y $(0,1,0)$ sobre la superficie $z=1-\sqrt{x^{2}+y^{2}}$ (\textbf{sugerencia:} usar coordenadas \underline{cilíndricas}.)
	\end{enumerate} 
	
	\begin{enumerate}[label={{\textcolor{trueblue}{\textbf{Ej}. \arabic*.0}}}, start=5]
		\item A particle is subjected to the potential $V(x) = -Fx$ where $F$ is a constant. The particle travels from $x=0$ to $x=a$ in a time interval $t_{0}$. Assume the motion of the particle can be expressed in the form $x(t)=A + Bt + Ct^{2}$. Find the values of $A$, $B$, and $C$ such that the action is a minimum
	\end{enumerate} 
	
	\begin{enumerate}[label={{\textcolor{trueblue}{\textbf{Ej}. \arabic*.0}}}, start=12]
		\item The term \textit{generalized mechanics} has come to designate a variety of classical mechanics in which the Langrangian contains time derivatives of $q_{i}$ higher than the first. Problems for which $\dddot{x} = f(x,\dot{x}, \ddot{x}, t)$ have been referred to as \textbf{jerky} mechanics. Such equations of motions have interesting applications in chaos theory (cf. Chapter 11). By applying the methods of the calculus of variations, show that if there is a Lagrangian of the form $L\left( q_{i}, \dot{q}_{i}, \ddot{q}_{i}, t \right)$, and Hamilton´s principle holds with the zero variation of both $q_{i}$ and $\dot{q}_{i}$ at the end points, then corresponding Euler-Lagrange equations are:
		
		\[  \frac{ d^{2} }{dt} \left(  \frac{ \partial L }{ \partial \ddot{q}_{i} } \right) - \frac{d}{dt} \left(  \frac{ \partial L  }{\partial \dot{q}_{i}   }  \right) + \frac{\partial L}{\partial q_{i}} =0 \quad i = 1,2,...,n  \]
		
		Apply this result to the Langranian:
		
		\[ L = - \frac{m}{2}q\ddot{q}-\frac{k}{2}q^{2} \]
		
		Do you recognize the equations of motion?
	\end{enumerate} 
	
\begin{enumerate}[label={{\textcolor{trueblue}{\textbf{Ej}. \arabic*.0}}}, start=14]
		\item A uniform hoop of mass $m$ and radius $r$ rolls without slipping on a fixed cylinder of radius $R$ as shown in the figure. The only external force is that of gravity. If the smaller cylinder starts rolling from rest on top of the bigger cylinder, use the method of Lagrange multipliers to find the point at which the hoop falls off the cylinder.
\end{enumerate} 



\begin{enumerate}[label={{\textcolor{trueblue}{\textbf{Ej}. \arabic*.0}}}, start=16]
	\item In certain situations, particularly one-dimensional systems, it is possible to incorporate frictional effects without introducing the dissipation function. As an example, find the equations of motion for the Lagrangian
	
	\[  L = e^{\gamma{t}} \left(  \frac{ m\dot{q}^{2}  }{2} - \frac{ kq^{2}  }{ 2  }      \right)        \].
	
	How would you describe the  system? Are there any constants of motion? Suppose a point transformation is made of the form
	
	\[ s = e^{\frac{\gamma{t}}{2} }q \]
	
	What is the effective Lagrangian in terms of $s$? Find the equation of motion for $s$. What do these results say about the conserved quantities for the system?  
	
	
\end{enumerate}
	
\end{multicols} 

\newpage


\tableofcontents \newpage

\begin{center}
	\section*{\textcolor{Cinnabar}{\bfseries{Soluciones}}}
	\addcontentsline{toc}{section}{\textcolor{Cinnabar}{\bfseries{Soluciones}} \hspace{1mm} \textcolor{bittersweet}{$\bullet$} \textcolor{Apple Green}{$\bullet$} \textcolor{trueblue}{$\bullet$} \textcolor{Aguamarina}{$\bullet$} \textcolor{Sun}{$\bullet$} \textcolor{orange(colorwheel)}{$\bullet$} \textcolor{Pesto}{$\bullet$} \textcolor{blue-violet}{$\bullet$} \textcolor{ticklemepink}{$\bullet$} \textcolor{Gallery}{$\bullet$} \textcolor{Mantle}{$\bullet$} \textcolor{Nevada}{$\bullet$} \textcolor{onyx}{$\bullet$}  } 
\end{center}  \vspace{0.5cm}


\begin{center}
	\begin{tikzpicture}
		% Dibuja círculos rellenos en diferentes colores
		\fill[bittersweet] (0,0) circle (0.17cm);
		\fill[Apple Green] (0.6,0) circle (0.17cm);
		\fill[trueblue] (1.2,0) circle (0.17cm);
		\fill[Aguamarina] (1.8,0) circle (0.17cm);
		\fill[Sun] (2.4,0) circle (0.17cm);
		\fill[orange(colorwheel)] (3.0,0) circle (0.17cm);
		\fill[Pesto] (3.6,0) circle (0.17cm);
		\fill[blue-violet] (4.2,0) circle (0.17cm);
		\fill[ticklemepink] (4.8,0) circle (0.17cm);
		\fill[Gallery] (5.4,0) circle (0.17cm);
		\fill[Mantle] (6.0,0) circle (0.17cm);
		\fill[Nevada] (6.6,0) circle (0.17cm);
		\fill[onyx] (7.2,0) circle (0.17cm);
		% Agrega más círculos o modifica los colores según sea necesario
	\end{tikzpicture}
\end{center} \vfill
\section{¿Qué es un texto científico?} \vspace{0.5cm}

Un \textbf{texto científico} es una producción escrita que comunica resultados de investigación, teorías o revisiones en un área del conocimiento. Puede presentarse en diversos formatos, entre los que destacan:

\begin{itemize}
	\item Artículo científico
	\item Texto académico
	\item Póster científico
\end{itemize} \vspace{0.5cm}

Ejemplo: La \textit{Revista Mexicana de Física}, actualmente en su versión 40.

\subsection{Tipos de publicaciones científicas} \vspace{0.5cm}

\begin{enumerate}
	\item Libros
	\item Artículos de revistas indexadas o arbitradas (impresos y electrónicos)
	\item Patentes (protegidas legalmente)
	\item \textit{Proceedings} de congresos (mini-artículos publicados en un volumen de congreso)
	\item Informes técnicos
	\item Páginas web científicas de organismos oficiales
	\begin{itemize}
		\item OMS
		\item Textos basados en investigación científica
	\end{itemize}
	\item Pósters de investigación
\end{enumerate} \vspace{0.5cm}

\subsection{Fuentes primarias y secundarias} \vspace{0.5cm}

En investigación, es importante distinguir entre \textbf{fuentes primarias} y \textbf{fuentes secundarias}:

\paragraph{Fuentes primarias:}  
Documentos originales que presentan resultados directos de una investigación.
\begin{itemize}
	\item Libros
	\item Artículos de revistas indexadas o arbitradas
	\item Bases de datos
	\item Patentes
	\item Páginas web científicas
\end{itemize}

\paragraph{Fuentes secundarias:}  
Materiales que recopilan, interpretan o resumen información de las fuentes primarias.
\begin{itemize}
	\item Tesis publicadas
	\item Proceedings de congresos
	\item Ponencias de congresos
	\item Informes técnicos y monografías (incluidos en web o blogs)
	\item Vídeos en línea
	\item Artículos de divulgación
\end{itemize}

\textbf{Nota:} En una tesis no se recomienda citar otra tesis como fuente principal. Un artículo de investigación tiene mayor valor académico que una tesis de licenciatura.

\subsubsection{En este taller se revisarán:}
\begin{enumerate}
	\item Póster
	\item Presentación académica
	\item Resumen de congreso
	\item Artículos de revistas indexadas o arbitradas
	\item Tesis
\end{enumerate}

\textbf{Sugerencia:} Los datos valiosos para una publicación deben protegerse antes de ser difundidos para evitar posibles robos de información. \vspace{0.5cm}

\section{Búsqueda de fuentes de información} \vspace{0.5cm}

\subsection{Bibliotecas BUAP} \vspace{0.5cm}

Las bibliotecas de la BUAP ofrecen diversos recursos para la investigación:

\begin{itemize}
	\item \textbf{Tesis impresas:} Son requisito académico. Están disponibles en el acervo o repositorio de acceso abierto. Se proporciona un formato para subirlas.
	\item \textbf{Colecciones:} Libros y revistas editados periódicamente.
	\item \textbf{Biblioteca digital:} Contiene bases de datos de distintas áreas (sociales, naturales, económicas, ciencias y tecnologías, entre otras).
\end{itemize}

\subsection{Uso ético de la información y Turnitin} \vspace{0.5cm}

\textcolor{Cerulean}{\textbf{Turnitin}} es una herramienta para verificar la originalidad de documentos académicos.  
En BUAP se recomienda un \textbf{mínimo del 30\% de originalidad} para evitar plagio. Turnitin puede detectar:
\begin{itemize}
	\item Patrones de plagio.
	\item Uso de inteligencia artificial y modelos de lenguaje (LLMs).
\end{itemize}

\paragraph{Reglas para evitar plagio:}
\begin{itemize}
	\item Referenciar siempre cualquier material ajeno: gráficas, figuras, fragmentos de texto, tablas, etc.
	\item Si se cita textualmente, colocar el texto entre comillas y agregar la referencia correspondiente.
	\item Aunque en una tesis a veces no se solicite permiso explícito para usar figuras, siempre se debe acreditar la fuente.
\end{itemize}

\textbf{Nota:} Tomar resultados de otra persona sin atribución, y que esto se descubra, puede dañar irreversiblemente la carrera científica. La ética es fundamental en la investigación.

\subsection{Interfaz de bibliotecas BUAP} \vspace{0.5cm}

Principales herramientas de búsqueda:
\begin{itemize}
	\item \textbf{Web of Science (WOS):} Motor de búsqueda de artículos científicos de diversas disciplinas. Permite refinar las consultas mediante palabras clave relevantes.
	\item \textbf{Google Académico / Google Scholar:} Plataforma de acceso abierto para buscar artículos, tesis y documentos académicos.
	\item \textbf{Open Knowledge Maps:} Herramienta que utiliza inteligencia artificial para mapear y explorar temas de investigación.
\end{itemize}

\subsection{Organización de clases} \vspace{0.5cm}
\begin{itemize}
	\item Martes y viernes: clases regulares.
	\item Jueves: revisión de actividades de talleres (posible realización en sala FM9).
\end{itemize}

\subsection{Tarea} \vspace{0.5cm}

\begin{enumerate}
	\item Realizar una búsqueda de tesis o tema de tesis en la base de datos.
	\item Aprender a realizar búsquedas académicas propias.
	\item Enviar la búsqueda realizada a \href{mailto:cmendoza@fcfm.buap.mx}{cmendoza@fcfm.buap.mx} antes del día jueves.
	\item Seleccionar \textbf{5 artículos relevantes} sobre el tema de interés.
	\item Entregar un documento (Word, LaTeX o PDF) que incluya:
	\begin{itemize}
		\item Nombre del estudiante.
		\item Línea de investigación actual o deseada.
		\item Enlace de cada artículo.
		\item Síntesis breve (2 líneas) de cada artículo, revisando resumen y resultados.
	\end{itemize}
\end{enumerate}


\end{document}
