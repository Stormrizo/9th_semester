\documentclass{article}
\usepackage{tikz}
\usetikzlibrary{angles, quotes}
\usepackage{amssymb}
\usepackage{pgfplots}
\pgfplotsset{width=10cm,compat=1.9}
\usepgfplotslibrary{statistics}
\usepackage[left=2.54cm,right=2.54cm,top=2.54cm,bottom=2.54cm]{geometry}
\usepackage{tasks} %Paquete necesario para la producción de items horizontales para algunos ejercicios de tarea empleando el comando /begin{multicols}}{#}.../end{multicols} para ayudarnos a reducir las páginas
\usepackage{pdflscape} %comando que permite cambiar a orientación horizontal las hojas%
\usepackage{soulutf8} %Paquete para permitir la compilación de acentos usando el comando {\hl{}}
\usepackage{soul} %Paquete para poder subrayar párrafos
\sethlcolor{yellow(munsell)} %comando para definir el color para el subrayado usando el comando \hl
\usepackage[utf8]{inputenc}
\usepackage{booktabs}
\usepackage{siunitx} % para unidades como g, mL, etc.
\usepackage[spanish]{babel}
\usepackage{icomma}
\usepackage{siunitx}
\usepackage{url}
\usepackage[colorlinks=true, urlcolor=blue,  linkcolor=black, citecolor=green]{hyperref}
\usepackage{pdfpages}
\usepackage{blindtext} %Paquete que genera texto ficticio (dummy text) con el comando: \blindtext
\usepackage{cancel} %in the preamble gives you four different modes of striking through: \cancel{text to cancel} draws a diagonal line (slash) through its argument, \bcancel{text to cancel} uses the negative slope (a backslash), \xcancel{text to cancel} draws an X (actually \cancel plus \bcancel), \cancelto{〈value〉}{〈expression〉} draws a diagonal arrow through the 〈expression〉pointing to the 〈value〉 (math-mode only)
\usepackage{csquotes}
\usepackage{afterpage}
\usepackage{parskip} 
\usepackage{float}
\usepackage{enumitem}
\usepackage{multicol}%Paquete que permite la creación aislada de columnas de texto. De esta forma se reduce la cantidad de páginas en nuestro documento
\usepackage{lipsum} %Paquete que genera texto de relleno al igual que \usepackage{blindtext}, se usa con el comando \lipsum[#-#] los #(hashtags) delimitan el número de páginas que deseamos ocupar del paquete lipsum para usarlos como relleno. 
\newenvironment{Figura}
{\par\medskip\noindent\minipage{\linewidth}}
{\endminipage\par\medskip}
\usepackage{caption}
\usepackage[
backend=biber,
sorting=none,
url=true
]{biblatex} %bibliografía
\addbibresource{biblio.bib}
% Establecer el espacio entre las entradas de la bibliografía
\setlength{\bibitemsep}{1\baselineskip} % Puedes ajustar el valor según tus preferencias
\usepackage{amsmath,amsthm,amssymb,amsfonts}
\usepackage{pifont} %Permite la compilación del comando \xmark: es el símbolo de la tachita
\usepackage{mathtools} %permite la compilación de símbolos para matrices
\usepackage{empheq} %Paquete que se relaciona con \usepackage[most]{tcolorbox} para la creación de cuadros/cajas de colores para delimitar los resultados matemáticos o líneas de texto usando la línea de comando: \begin{empheq}[box={\mymath[colback=orange(webcolor)!70,drop lifted shadow]}]{equation*} \end{empheq}
\usepackage[most]{tcolorbox} %Paquete que se relaciona con \usepackage{empheq} para la creación de cuadros/cajas de colores para delimitar los resultados matemáticos o líneas de texto
\renewcommand{\qedsymbol}{$\blacksquare$}
\usetikzlibrary{positioning,decorations.pathreplacing} %Paquete que permite la compilación de llaves para cuadros sinópticos (brace diagram) 
\usepackage{schemata} %The schemata package is designed just for these brace diagrams

\newcommand\AB[2]{\schema{\schemabox{#1}}{\schemabox{#2}}} %comando o paquete necesario para crear cuadros sinópticos

\newtcbox{\mymath}[1][]{%
nobeforeafter, math upper, tcbox raise base,
enhanced, colframe=blue!30!black,
colback=blue!30, boxrule=1pt,
#1} %Comando importante para encasillar los resultados matemáticos en cajas de diferentes colores, se relaciona con los paquetes \usepackage{empheq} y \usepackage[most]{tcolorbox} 

\newenvironment{sysmatrix}[1]
{\left(\begin{array}{@{}#1@{}}}
{\end{array}\right)}
\newcommand{\ro}[1]{%
\xrightarrow{\mathmakebox[\rowidth]{#1}}%
}
\newlength{\rowidth}% row operation width
\AtBeginDocument{\setlength{\rowidth}{3em}}  %Comando importante para laproducción de lineas de operaciones entre matrices, método de Gauss o Gauss Jordan se relaciona con \newenvironment{sysmatrix}

%formato para cambiar el horario a español
\usepackage[spanish]{datetime2}
\DTMsetdatestyle{spanish}

\renewcommand{\today}{\DTMdisplaydate{\the\year}{\the\month}{\the\day }{-1}}
%%%%%%%%%%%%%%%%%%%%%%%%%%%%%%%%%%%%%%%%%

\usepackage{datetime}
\newcommand{\mycurrenttime}{\xxivtime}

%%%%%%%%%%%%%%%%%%%%%%%%%%%%%%%%%%%%%%%%%%%%%%%%%%%%%%%%%%%%%%%%%%%%%%%%%%%%%%%%%%%%%%%%%%%%%%%%%%%%%%%%%%%%%%%%%%%%%%%%%%%%%%%%%%%%%%%%%%%%%%%%%


%%%%%%%%%%%%%Caja de color para el título%%%%%%%%%%%%%%%%%%%%%%%%%%%%%%%%%%%%%%%%%%%%%%%%%%%%%%

\definecolor{myframecolor}{RGB}{85, 100, 19} % Saratoga
\definecolor{myboxcolor}{RGB}{177, 196, 56} % Earls Green

\newtcolorbox{mybox}{
enhanced,
colback=myboxcolor!12, % Color de fondo del cuadro
colframe=myframecolor, % Color del marco del cuadro
arc=0pt,
boxrule=1pt,
borderline west={2mm}{-10mm}{myframecolor}, % Borde en el lado izquierdo
sharp corners=southwest,
width=\linewidth,
before=\par\vspace{\bigskipamount}, % Espacio antes del cuadro
after=\par\vspace{\bigskipamount} % Espacio después del cuadro
}


\newcommand{\euler}{e} %Comando para producir letra e de euler
\newenvironment{remark}{\par\vfill\footnotesize % Vertical white space above the remark and smaller font size

\begin{list}{}{
		\leftmargin=80pt % Indentation on the left
		\rightmargin=60pt}\item\ignorespaces % Indentation on the right
	\makebox[-2.5pt]{\begin{tikzpicture}[overlay]
			\node[draw=Horizon!60,line width=2.5pt,circle,fill=Horizon!25,font=\sffamily\bfseries,inner sep=4pt,outer sep=0pt] at (-19pt,5pt){\textcolor{Horizon}{Nota}};\end{tikzpicture}} % Blue Nota in a circle
	\advance\baselineskip -1pt}{\end{list}\vskip5pt} % Tighter line spacing and white space after remark
	\usepackage{graphicx}
	
	\usepackage{titling}
	
	\input{colores}
	
	%%Producir signos de cita textual``Asignatura''
	
	\newcommand{\xmark}{\ding{55}} %%%comando que genera la tachita
	
	\newenvironment{MyColorPar}[1]{%
\leavevmode\color{#1}\ignorespaces%
}{%
}%

%%%%%%%%%%%%%%%%%%% Título de la tarea, Nombre de alumno

\title{Espectroscopía UV-VIS de muestras de cafeína}
\author{\emph{Espectrómetro} $\boldsymbol{\mid}$ Instrucciones}
\date{\today}

...

\section{Pasos} \vspace{0.5cm}

\begin{enumerate}
	\item Prender el espectrómetro:
	\begin{itemize}
		\item[a)] Monitor (pantalla y caja)
		\item[b)] Espectrómetro
	\end{itemize}
	
	\item Establecer \underline{parámetros}\footnote{
	\textbf{Data mode}: \emph{Absorbance}, 
	$\lambda_{start}$ = 200 nm, $\lambda_{stop}$ = 350 nm,
	\textbf{Data interval}: Quant, \textbf{Lamp Change}: 340 nm, \textbf{Smoothing}: Medium.
	} para absorbancia.
	
	\item Línea base con agua desionizada (cubeta de cuarzo limpia).
	\item Medir estándares y muestras diluidas de cafeína.
	\item Guardar datos.
\end{enumerate}

---

\section{Preparación de soluciones estándar de cafeína}

\begin{enumerate}
	\item \textbf{Solución madre 1000 mg/L}: disolver 100 mg de cafeína pura en 100 mL de agua desionizada. Filtrar (0.45 μm).
	\item \textbf{Solución intermedia 100 mg/L}: tomar 10.0 mL de la madre y aforar a 100 mL con agua.
	\item Preparar estándares de 5, 10, 20, 30, 40 y 50 mg/L a partir de la intermedia usando diluciones volumétricas.
	\item Preparar un blanco con agua desionizada.
\end{enumerate}

---

\section{Preparación de muestras}

\subsection*{Polvo de cafeína}
\begin{itemize}
	\item Pesar $\approx$50 mg, disolver en 50 mL de agua, filtrar.
	\item Diluir para que la concentración quede en el rango de la curva.
\end{itemize}

\subsection*{Tabletas de 200 mg}
\begin{itemize}
	\item Triturar la tableta en mortero.
	\item Disolver en $\approx$100 mL de agua a 40 °C, enfriar y filtrar.
	\item Diluir para entrar en el rango lineal de la curva.
\end{itemize}

---

\section{Medición y análisis}
\begin{enumerate}
	\item Medir absorbancia de todos los estándares a $\lambda$ = 273–275 nm.
	\item Construir la curva de calibración (A vs concentración).
	\item Medir muestras y calcular concentración usando la ecuación de la recta.
\end{enumerate}

---

\begin{table}[H]
	\centering
	\caption{Ejemplo de datos de calibración}
	\label{tabla:pruebas_cafeina}
	\begin{tabular}{@{}lcc@{}}
		\toprule
		Concentración (mg/L) & Absorbancia & Observaciones \\
		\midrule
		0  & 0.000 & Blanco \\
		5  & 0.105 & \\
		10 & 0.210 & \\
		20 & 0.420 & \\
		30 & 0.625 & \\
		40 & 0.840 & \\
		50 & 1.050 & Límite alto lineal \\
		\bottomrule
	\end{tabular}
\end{table}

---

\section*{Notas}
\begin{itemize}
	\item La cafeína presenta un $\lambda_{max}$ cercano a 274 nm en agua.
	\item La solubilidad a 25 °C es $\approx$21.6 mg/mL, por lo que no se requieren disolventes orgánicos.
	\item Mantener las cubetas limpias y libres de burbujas.
\end{itemize}


Se pesó 0.0264g de cafeína, es decir 26.9mg 

660 mg x L <- concentración de la muestra madre 

\printbibliography[heading=bibintoc]





\end{document}
